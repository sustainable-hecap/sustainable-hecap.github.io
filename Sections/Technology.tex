\documentclass[../SustainableHEP.tex]{subfiles}
\graphicspath{{\subfix{Sections/Figs/}}}
\begin{document}
\RaggedRight
\sloppy
\newpage

%%%%%%%%%%%%%%%%%%%%%%%%%%%%%%%%%%%%%%%%%%%%%%%%%%

\section{Research Infrastructure and Technology}
\label{sec:Technology}
%\textbf{Contributors:} Frauke Poblotzki, Karolos Potamianos,
%Jacopo Ghiglieri\\

%%%%%%%%%%%%%%%%%%%%%%%%%%%%%%%%%%%%%%%%%%%%%%%%%%

\begin{center}
    \includegraphics[width=\SDGsize]{Common/SDG_6_CleanWater.png}~%
    \includegraphics[width=\SDGsize]{Common/SDG_7_CleanEnergy.png}~%
    \includegraphics[width=\SDGsize]{Common/SDG_9_IndustryInnovation.png}~%
    \includegraphics[width=\SDGsize]{Common/SDG_11_SustainableCities.png}~%
    \includegraphics[width=\SDGsize]{Common/SDG_12_ResponsibleConsumption.png}~%
    \includegraphics[width=\SDGsize]{Common/SDG_13_ClimateAction.png}~%
    \includegraphics[width=\SDGsize]{Common/SDG_15_LifeOnLand.png}
\end{center}

%%%%%%%%%%%%%%%%%%%%%%%%%%%%%%%%%%%%%%%%%%%%%%%%%%

\exSum

\noindent A major aspect of sustainability in \ACR\ is the research infrastructure itself. The accelerators, experiments, observatories, setups and buildings have a lifetime impact from cradle to grave. This is recognised in section 8 of Sustainability Considerations of the European Strategy for Particle Physics~\cite{EuropStrategyPP}.
It divides the topic into three aspects: Energy efficient technologies, Energy efficient accelerator concepts and General sustainability aspects. The first two focus on the biggest impact of accelerators:\ the energy consumption during their operation. One aspect focuses on the current technology and its energy efficiency, the other one on the development of new accelerating concepts with smaller energy requirements. These topics are discussed more in depth in other sections of the Strategy. The third one broadens the view to an overall sustainability beyond energy:

\begin{quotation}
    "A carbon footprint analysis in the design phase of a new facility can help to optimize energy consumption for construction and operation. For cooling purposes accelerator facilities typically have significant water consumption. Cooling systems can be optimized to minimize the impact on the environment. For the construction of a facility environment-friendly materials should be identified and used preferably. The mining of certain materials, in particular rare earths, takes place in some countries under precarious conditions. It is desirable to introduce and comply with certification of the sources of such materials for industrial applications, including the construction of accelerators. A thoughtful life-cycle management of components will minimize waste."
\end{quotation}

\clearpage
\begin{reco2}{\currentname}
{
\begin{itemize}[leftmargin=6 mm]
\item Seek out new innovations and best practice.

\item Rethink how the impact of frequently-used equipment can be reduced, and reduce "over-design" by reassessing safety factors and other margins to reduce resource consumption.

\item Read section on resources and waste (Section~\ref{sec:Waste}).

\end{itemize}
}
{
\begin{itemize}[leftmargin=6 mm]

\item  Ensure that environmental sustainability is an essential consideration at all stages of projects, from initial proposal, design, review and approval, to assembly, commissioning, operation, maintenance, decommissioning and removal, using Life-Cycle Assessment and related tools.

%\item  Prioritise environmental sustainability in all aspects of the design and operational phases of projects, including Life-cycle-Assessment in any Scope.

\item Engage with industrial partners who exemplify best practice and sustainable approaches.

\item Appoint a dedicated sustainability officer to oversee project development, and institute regular meetings with a focus on environmental sustainability.
% \begin{itemize}
%     \item Have a specialised meeting to think about the environmental impact of the project you work on.
%     \item Make a Life-cycle-Assessment in any Scope you can for your project to make solutions comparable.
% \end{itemize}
\end{itemize}
}
{
\begin{itemize}[leftmargin=6 mm]

\item Critically assess the environmental impact of materials, construction and the operational lifecycle as an integral part of the design phase for all new infrastructure.

\item Provide training opportunities, required tools and technical support to assess and improve the environmental sustainability of project life-cycles.

\item Recognise and reward innovations that minimize negative environmental impacts, regardless of revenue.

\item Promote knowledge exchange on sustainability initiatives between groups and institutions, including decision-makers, designers and operators of projects, setups and infrastructure.

% \begin{itemize}
%             \item Require risk analysis or Life-Cycle Assessment for new developments and existing setups, to include not only safety but also environmental aspects.
%             \item 
%             \item Establish an exchange with initiatives and other institutes, which includes the designers, operators and decision-makers of projects, setups and infrastructure.
%             \item Propagate the transfer of knowledge between groups and the expansion of knowledge of technical staff.
%             \item Provide designers and operators with easy to use tools to see if an improvement has financial and sustainable advantages.
% \end{itemize}
\end{itemize}
}

\end{reco2}

%%%%%%%%%%%%%%%%%%%%%%%%%%%%%%%%%%%%%%%%%%%%%%%%%%


\subsection{Sustainability initiatives in technology}
The paths to sustainably improve research infrastructure and technologies are as manifold as the technical challenges. Some more general initiatives can be found in the subsection {Previous and Parallel Initiatives}. Some specific initiatives have been formed to help with the transfer of knowledge and technology within the community.

\begin{itemize}

    \item The International Committee for Future Accelerators (ICFA) has the specific panel “Sustainable Accelerators and Colliders”~\cite{SustainableAcceleratorsICFA}
    
    \item Every 2 years since 2011, the ESSRI workshop (Energy for Sustainable Science at Research Infrastructures~\cite{ESSRI5}) takes place.
    
    \item Innovation Fostering in Accelerator Science and Technology (I.FAST)~\cite{IFAST} is an EU-project in which the “WP 11 – Sustainable concepts and technologies” is aimed to increase sustainability. The current participating institutes are CERN, DESY in Germany, ESS in Sweden, GSI in Germany, PSI in Switzerland and STFC in the United Kingdom.

\end{itemize}


\subsection{The methodology of Life-Cycle Assessment}

The methodology of a life-cycle assessment can be used to analyse the environmental impact of resources used to build, run and decommission an accelerator, observatory, or experiment.  For more information see Section~\ref{sec:sustainablesourcing}.  Such calculations to estimate a variety of impacts have already been attempted by the following facilities:
\begin{itemize}
    \item The European Southern Observatory (\acrshort{eso})~\cite{ESO}.
    \item The GRAND Project, a multi-decade astrophysics experiment~\cite{Aujoux_2021} --- This led to a full issue of the Nature Astronomy Journal on climate change~\cite{NatureClimateIssue}.
    \item The RUEDI facility at STFC Daresbury Laboratory ~\cite{Shepard}.
    \item The Compact Linear Collider (\acrshort{clic}) is planning to conduct an assessment ~\cite{privateBList}.
    \item The ISIS-II project is planning to conduct and assessment ~\cite{ISISII}.
\end{itemize}

As an example for a particle detector, silicon wafers are one of the materials used. In the Life-Cycle assessment this would be one of the input materials. The analysis takes into account the respective ecological impacts of this wafer from a database. The data of a 1 ${\rm cm}^2$ silicon wafer (thickness = 775 $\mu$m, diameter = 300 mm, weight = 0.128 kg) as identified in 2000 is given in Tables~\ref{tab:InputOutput} and \ref{tab:InputOutput2}~\cite{ProBasSi}:

%%%%%


\begin{table}
\captionsetup{type=table}
\caption[Inputs \& outputs of silicon wafer production]{Inputs \& Outputs of silicon wafer production.~\cite{ProBasSi}}
\label{tab:InputOutput}
\resizebox{\textwidth}{!}{
\begin{tabular}{m{0.3\textwidth}m{0.2\textwidth}m{0.3\textwidth}m{0.2\textwidth}}

\toprule
Inputs&
Quantity&
Outputs&
Quantity\\
\midrule
Hydrogen chloride HCl (hydrochloric acid)&
0.00675 kg&
Co-products: Si in other co-products&
0.000286 kg\\
\midrule
Graphite (as electrode material)&
0.000163 kg&
Co-products: Silicon tetrachloride&
0.00415 kg\\
\midrule
Wood chips&
0.00183 kg&
Co-products: Si residues for solar cells&
65.2$\times$10-6 kg\\
\midrule
Petroleum coke&
0.000597 kg&
Polished silicon wafer&
1 cm2\\
\midrule
Quartz&
0.00487 kg&
&
\\
\midrule
Electricity&
0.385 kWh&
&
\\
\midrule
Dry wood&
0.00398 kg&
&
\\
\bottomrule
\end{tabular}
}
\end{table}

%%%%%

\begin{table}
\captionsetup{type=table}
\caption[Emissions of silicon wafer production]{Emissions of silicon wafer production.~\cite{ProBasSi}}
\label{tab:InputOutput2}
\resizebox{\textwidth}{!}{
\begin{tabular}{m{0.25\textwidth}m{0.25\textwidth}m{0.25\textwidth}m{0.25\textwidth}}
\toprule
Air emissions&
Quantity&
Water discharge&
Quantity\\
\midrule
CH${}_4$&
68.8$\times$10-6 kg&
Metal chlorides&
0.000787 n/a kg\\
\midrule
CO&
0.000167 kg&
& 
\\
\midrule
CO${}_2$&
0.00833 kg&
Waste&
Quantity\\
\midrule
Ethane&
29$\times$10-6 kg&
SiO${}_2$&
16.3$\times$10-6 kg\\
\midrule
H${}_2$O&
0.00188 kg&
&
\\
\midrule
Methanol&
85.1$\times$10-6 kg&
&
\\
\midrule
NOx&
13.8$\times$10-6 kg&
&
\\
\midrule
Particulate matter&
0.000201 kg&
&
\\
\midrule
SO${}_2$&
34.4$\times$10-6 kg&
&
\\
\midrule
Hydrogen&
0.000125 kg&
&
\\
\bottomrule
\end{tabular}
}
\end{table}

To estimate the impact and the sustainability of the technology specific to \ACR/  (e.g. an accelerator magnet), the reporting through a Life-Cycle Assessment is mandatory. As a start, this could include a standardized way to report the energy consumption of equipment and machines during operation, but to do a more complete Life-Cycle Assessment it would include e.g. the use of rare earths in the magnet fabrication.

Proof that some optimization comparing to previously used equipment has taken place could be an option, too. Publications as references with this kind of data should become part of a database to ease comparability and implementation of more advanced technology in the field.

The topic of use of materials is also part of the section about Waste (\sref{sec:Waste}).


\begin{casestudy}[\label{case:FutureColliders}Sustainability of Future Colliders]{Sustainability of Future Colliders}%
\noindent The future of high energy physics includes decisions on Future Collider Facilities to be built. Figure~\ref{fig:powerCollider} compares the energy needs of Future e+e- Colliders. The projected grid power during operation is given, including parts for the laboratory, computer center, detector, etc.

\begin{figure}
    \includegraphics[width=.8\textwidth]{Technology/power_vs_logE_withLHCandCERN.png}
    \caption{Site power required for proposed $e^+e^-$ collider projects at different center-of-mass energies, compared to the power consumption of CERN and LHC today. ~\cite{privateJList} }\label{fig:powerCollider}
    %[1]  private communication J. List 
\end{figure}


The sustainability aspects of future facilities will become an increasingly important factor in the decision process about which, if any, facility should be built.
To evaluate the \CdO\ footprint of the electricity consumption of a future colliders and therefore its impact is complex. The assumptions such as "in 2045, electricity will be \CdO\ neutral anyway" or "only buy green electricity, therefore the \CdO\ impact of the used electricity is zero" are debatable. To use the currently valid \CdO\ footprint for electrical energy is of course not entirely correct either, as this will most likely change by the time operation of the facilities begins.
Additionally, the argumentation of "electrical effectiveness" should be discussed. What are scientifically relevant benchmarks and which ones make sense, e.g. kWh/Higgs or kWh/luminosity compared to the required connected power?
It could therefore be important to plan these facilities in such a way that they can adapt at least some of their power consumption to the real availability of renewable energy. 
% Pulsed accelerators have some advantage over storage rings with complicated fill pattern/top-up etc.

Thinking about the life cycle of such facilities is another complex subject. Not only the accelerator itself, but also the tunnel and the entire infrastructure. The impact of construction, then deconstruction (except for the tunnel) and disposal (including activated materials, which may constitute a radiation hazard) after a few years is not negligible.
\end{casestudy}

%%%%%%%%%%%%%%%%%%%%%%%%%%%%%%%%%%%%%%%%%%%%%%%%%%

\subsection{Examples of technology improvements}

In the last years, developments have been made to improve the use of resources for experiments, accelerators, facilities and setups. In the following some of these shall be presented.

%\begin{itemize}
%    \item Energy Recovery Linacs, such as the Powerful Energy Recovery Linac for Experiments (PERLE)~\cite{PERLECDR}, which provides an electron beam of approximately 1~GeV energy, and the Cornell-BNL ERL Test Accelerator (CBETA)~\cite{CBETACDR}.  
%\end{itemize}

\begin{bestpractice}[\label{BP:EnergyRecoveryAccelerator}Realization of a multi-turn energy-recovery accelerator]\\
The operation of particle accelerator facilities is inherently resource-intensive, and thus poses a challenge to sustainability. In line with acknowledging our responsibility for sustainable usage of energy resources, the development, establishment, and demonstration of a scalable multi-turn Energy Recovery Linac (ERL) with efficient energy recycling was implemented at the S-DALINAC accelerator at TU Darmstadt, Germany~\cite{Arnold:2020snn}. An efficient energy-recycling in multi-turn operation with a saving of up to 87 \% of the beam power-consumption in the main LINAC has been recently demonstrated. This result, together with further developments on multi-turn ERLs is a promising basis for future high-power beams that truly support sustainability aspects. (ER@CEBAF, USA~\cite{Meot:2018yoo}; MESA ERL, Germany~\cite{MESA}; International PERLE Collaboration~\cite{PERLE, PERLECDR}; CBETA the Cornell-BNL ERL Test Accelerator ~\cite{CBETACDR}; for an overview, see~\cite{Klein:2022lgx, Hutton:2022kac}) 
\end{bestpractice}

\begin{casestudy}[Sustainability of plasma wakefield acceleration technology for future accelerators\label{case:wakefield}]\\
A promising technology which would reduce both the material and energy cost of accelerating particles, and hence improve its environmental sustainability, is wakefield acceleration.\cite{wakefield1}  These use laser pulses (in the case of laser wakefield accelerators), or particle beam bunches (for plasma wakefield accelerators, or PWFAs) as the driver to accelerate plasma electrons, creating ion cavities.  This creates an electric field that pulls electrons back to their original positions, which they overshoot, creating waves in the plasma, known as the wakefield.  These plasma waves can accelerate electrons by transferring energy from the drive beam to electrons by putting electrons just behind the drive beam.  Wakefield technology routinely gives acceleration gains of $10-100$GeV/m \cite{wakefield1}, thus resulting in a significant reduction of the resources required (materials, energy) to build particle accelerators.  For example, accelerating electrons to 1 TeV of energy using PWFA would require \eg only a 21 km-long particle accelerator, while CLIC technology needs 52 km~\cite{wakefield2}.  Furthermore there are indications that PWFA is more power efficient at high energies than conventional accelerator technology~\cite{wakefield2}.
\end{casestudy}


\begin{casestudy}[LHCb and sustainability\label{case:LHCb}]\\
{\footnotesize Edited from Framework TDR for the LHCb Upgrade II~\cite{LHCbU2FTDR}: Opportunities in flavour physics, and beyond, in the HL-LHC era, edited extracts from the U2 FDTR chapter on environmental impacts of the project, as contributed by Chris Parkes.}%]{LHCb and susainability}%

In a world with increasing demand on limited resources and undergoing climate change, the LHCb collaboration feels a responsibility to consider energy consumption, sustainability and efficiency when discussing our scientific proposals. To this end the Framework Technical Design Report of the next-generation LHCb Upgrade II experiment~\cite{LHCbU2FTDR} has included a dedicated chapter on these considerations analysing the current Upgrade I system and indicating directions for future investigation. This section reports some of the main elements.  

The 2020 update of the European Strategy for Particle Physics~\cite{EuropeanStrategy2020} reports: ``The environmental impact of particle physics activities should continue to be carefully studied and minimised. A detailed plan for the minimisation of environmental impact and for the saving and re-use of energy should be part of the approval process for any major project. Alternatives to travel should be explored and encouraged.'' 
As one of the major experimental infrastructures operating at the LHC,  our environmental protection strategy should be made in coordination with CERN guidelines, as described in the first CERN environment report~\cite{envrep2020}.

CERN has a formal objective to reduce direct emissions (``Scope 1'') by $28\%$ by the end of 2024. These are dominated by the activities of the LHC experiments, and in particular by the use of fluorinated gases for particle detection and detector cooling purposes, as shown in Fig.~\ref{fig:cern_co2}. These emissions have to be carefully considered in the operation of the Upgrade I detector and in the design of its future upgrade. Other relevant aspects of the environmental impact of our project are the power consumption of the experimental infrastructure (indirect emissions, ``Scope 2''), the impact of digital technologies, and travel of the members of the collaboration. Fig.~\ref{fig:lhcb_co2} shows the relative contribution of each of these sources to the CO$_2$ equivalent footprint of the experiment operations expected during Run 3. %}

\begin{figure}
    \includegraphics[width=.8\textwidth]{Technology/cern_co2.png}
    \caption[CERN Scope 1 (direct) and Scope 2 (indirect, by electricity consumption) emissions for 2017 and 2018]{CERN Scope 1 (direct) and Scope 2 (indirect, by electricity consumption) emissions for 2017 and 2018, in CO$_2$ equivalent tons, by category; ``other'' includes air conditioning, emergency generators and CERN vehicle fleet fuel consumption (reproduced from Ref.~\cite{envrep2020}).  `Budapest' refers to electricity use at the (now inactive) Wigner data centre in Hungary.}\label{fig:cern_co2}
\end{figure}

\begin{figure}
    \includegraphics[width=.8\textwidth]{Technology/LHCbEmissionsCo2ePie.png}
    \caption[Expected relative contribution to CO$_2$ equivalent emissions from LHCb operations in Run 3]{Expected relative contribution to CO$_2$ equivalent emissions from LHCb operations in Run 3. The total emissions are estimated to be 4400 tonnes CO$_2$ equivalent per annum.}\label{fig:lhcb_co2}
\end{figure}

\paragraph{Direct emissions}
% Chris liaising Heinrich, Eric, Carmelo

Direct emissions from LHCb are dominated by losses of gases with sizeable global warming potential (\acrshort{gwp}). The GWP is the heat absorbed by a greenhouse gas in the atmosphere as a multiple of the same mass of CO$_2$.  It thus allows conversion into CO$_2$ equivalent emissions. As some gases breakdown they have time-dependent values and we use the 100 year value~\cite{AR5}. The gases are utilised in LHCb in detector cooling systems and in the detection systems. Improvements made in the cooling systems of LHCb mean that emissions are now dominated by the detection system in Upgrade I. All systems are closed, with emissions being the result of losses.

In the original LHCb detector of Run 1 and 2
the gas C$_6$F$_{14}$ (GWP 7910) was used in cooling plants. For upgrade I ``Novec 649" (GWP 1) is planned to be used and 
increased use of low-impact CO$_2$ based cooling.
For Upgrade II, lower operating temperatures are foreseen and the GWP of the cooling systems will be considered.

In the detector systems the Ring Imaging Cherenkov Systems (RICH1 and 2) and Muon systems of Upgrade I use greenhouse gases. The RICH2 system currently uses CF$_4$ (GWP 6630) and RICH1 C$_4$F$_{10}$ (GWP 9200) radiators. R\&D will be pursued for Upgrade II on alternative gases, RICH2 is looking at CO$_2$ use where a test has already been performed, and leakless systems. Significant effort has been made to minimise leaks.  
In the original LHCb detector GEM detectors (gas electron multiplier detectors) were utilised in a part of the muon system, the removal of these for Upgrade I reduces the detector system emissions by 40\%. 
Recirculating systems are used throughout. The study of alternative gas mixture will be conducted to reduce the CF$_4$ consumption in the proposed future muon systems.

The CO$_2$ equivalent emissions expected in Run~3 are shown in figure~\ref{fig:LHCbCO2e}. These are taken from the average values of annual usage during Run 2 for the detector systems that are still present, or that have been replaced with similar systems for Run~3. 

\begin{figure}
    \includegraphics[width=.8\textwidth]{Technology/lhcb_co2e.png}
    \caption[Expected Scope 1 (direct)  emissions in CO$_2$ equivalent tonnes from the LHCb detector gas systems in Run 3]{Expected Scope 1 (direct)  emissions in CO$_2$ equivalent tonnes from the LHCb detector gas systems in Run 3. The data is taken from the annual emissions of the systems, or predecessors, during Run 2.}\label{fig:LHCbCO2e}
\end{figure}

\paragraph{Power consumption}   % Matteo
\label{sec:powerconsumption}
CERN peak power demand, with the full accelerator chain running, is about $180$~MW, which brings the total annual energy consumption to $1.2$~TWh. This very large energy demand is partially mitigated by the fact that the electricity procurement is mainly from France, whose production capacity is 87.9$\%$ carbon-free (2017-figures).\footnote{It has been argued that french energy production is part of the common EU~market and that it would therefore be more adequate to use a conversion factor for an EU mix, which is about a factor of five higher than that for France~\cite{EUmix}.} This keeps the contribution from the electrical power to the total CERN emission budget below 20$\%$, as shown in Fig~\ref{fig:cern_co2}. Nevertheless, guided by the Energy Management Panel, EMP, CERN is spending a large effort to improve energy efficiency, with special focus on the accelerator sector. As an example, in the transition to the HL-LHC, with a tenfold increase in luminosity, the Organization’s immediate priority is to limit the increase in energy consumption to 5$\%$ up to the end of 2024.
 
LHCb during a normal data taking period of Run 2 had a peak power demand of stably around 5.5~MW, of which 4.6~MW was from the experiment dipole magnet and the rest from the detector electronics and the online computing farm.  The Run 3 expectation is for an increase of $\sim$1.5~MW due to the increased demand of data processing power, which is to be compared to the five-fold increase in luminosity.
For Run 5 and beyond, the contribution of online computing is expected to increase substantially, as a consequence of a further order of magnitude increase in the data throughput.

For the power dissipated by the LHCb magnet, an important mitigation has been implemented very recently by CERN with the installation of a heat-recovery plant at the experimental site. This is intended to use the
hot water produced by the magnet and the machine cooling systems to heat a new residential area in the
town of Ferney-Voltaire next to the LHCb site. Thanks to this project, up to 8000 people’s homes will be heated at a lower cost and with reduced CO$_2$ emissions, corresponding to $\sim$2.5\% of the total CERN emission budget per year. 

\paragraph{Digital technologies}

The power consumption of the online computing farm at LHCb has been about 530~kW on average during Run~2 of the LHC. To cope with the significantly increased computing needs after Upgrade~I, a new data centre has recently been installed at Point~8 and the power consumption for computing is going to increase to 2000~kW for the upcoming data taking periods. 
The new computing data centre at Point~8 is located in a surface building and for practical reasons could not be included in the heat recovery project discussed in Sec.~\ref{sec:powerconsumption} above. However, great care has been put into the design to optimize its power efficiency, for example by implementing a state-of-the-art indirect free air cooling system with adiabatic assist~\cite{lbldcreport}.
A \acrshort{pue} of better than 1.08 has been achieved for the new data centre at Point~8, a value that compares favourably with other large computing centres~\cite{pue-data}. 

While it does not seem feasible to further improve the PUE of the data centre, energy savings could potentially be achieved by adjusting the operating mode to the actual computing needs at a given point in time.  Significant improvements in energy efficiency can be achieved by rewriting software so that it can efficiently exploit today's highly parallel computing architectures. LHCb has been doing this in preparation for Run~3 datataking and the impact of these activities on the energy efficiency of our software has been documented in~\cite{hlt1-energy-eff}. In total the energy efficiency of HLT1 software has been improved by a factor $4.8\times$ on CPUs, with the improvements coming in roughly equal parts from physics optimizations and the rewrite of the underlying software framework. A further improvement in energy efficiency  can also be achieved by porting suitable algorithms from CPU to more efficient technologies such as GPU, FPGA or even custom-made ASICs. LHCb has demonstrated this with the Allen project~\cite{Allen}, which implemented HLT1 on GPUs, leading to an overall improvement in energy efficiency of up to $19\times$ compared to the Run~2 architecture. These improvements require significant effort and investment, above all in the training and retention of scientists able to effectively program across a range of modern computing architectures. 

The energy efficiency of the underlying computer hardware has also improved substantially over time. For example, the
AMD~7502~\cite{AMD7502} CPUs which were evaluated as candidates for LHCb's Run~3 HLT are $2.6\times$ more energy efficient than 
the benchmark E5-2630~Xeon CPUs used by LHCb during Run~2. 
Within a given computing architecture, energy savings can also be achieved by purchasing more expensive, higher quality hardware. 
As an example from the world of CPU, the more expensive AMD~7742~\cite{AMD7742}
provides twice the number of CPU cores and threads as the cheaper AMD~7502~\cite{AMD7502},
while its specified power consumption is only 25\% higher. The energy consumption and carbon footprint from data transfer, data storage and offline computing are much harder to assess than those for online computing, due to the distributed nature of the computing model with data centres and users distributed over many different countries. The GRAND collaboration has performed pioneering work in this direction~\cite{GRAND}.

\paragraph{Mobility}
%Jonas
As an international collaboration operating in an international field of research, travel is an intrinsic part of how LHCb operates. We have estimated the environmental impact of travel in order to attend LHCb collaboration meetings and international conferences. We have taken into account local commuter travel or travel related to on-site work at LHCb, such as shifts, although this is probably significant.

The impact of travel per participant for a typical LHCb collaboration week, pre-pandemic, corresponded to around $0.5\,\mathrm{tCO_2e}$ with the average LHCb week in 2019 leading to travel-emissions of $\sim$ 180$\,\mathrm{tCO_2e}$. The Speakers' Bureau database provides a complete record of all LHCb conference talks, allowing us to estimate the environmental impact in terms of   $\mathrm{tCO_2e}$ per year. LHCb weeks and conference travel contribute a total of approximately $1,000\,\mathrm{tCO_2e}$ per annum, a similar carbon footprint to the Run~3 experiment's projected electricity use due to online computing and the magnet (French energy mix). LHCb weeks contribute about three times as much to  LHCb's carbon footprint as conference travel.
 The carbon footprint of virtual conference attendance is calculated according to the life-cycle and operating costs of endpoint devices estimates in Ref.~\cite{VideoConCO2}, and is small.

LHCb, in common with other HEP collaborations, had extensive experience with virtual meetings before COVID, and videoconferencing
technology has already helped to reduce travel-related emissions over the past decade.
However the pandemic, as well as recent improvements to the videoconferencing software infrastructure, have shown us ways 
in which the organisation of virtual meetings can be improved and made more inclusive. 
At the same time the pandemic has also reminded us of the ongoing importance of in-person interaction,
not least to avoid fracturing the collaboration between those who can regularly travel to CERN in eco-friendly ways and those who cannot.
The collaboration has only just started to navigate this tension, but is actively exploring ways to reduce its 
travel-related environmental impact.

\end{casestudy}

\begin{bestpractice}
[Standardized accounting
of the carbon footprint of French research 
institutions: labos1point5 \label{BP:l1p5}]{Standardized accounting
of the carbon footprint of French research 
institutions: labos1point5}

\emph{Laboratoires de recherche}, loosely translated as research labs,
are the entities around which most of French research is organised.
They enjoy a relevant degree of autonomy, including 
aspects such as scientific goals and experimental designs. 
Access to research facilities, as well as 
a fraction of the annual budget is also managed at the lab scale.
Hence, this makes it a relevant scale to tackle the question 
of the carbon footprint of academic research. 

This has motivated the creation of the \emph{labos1point5} 
working group (\emph{Groupement De Recherche}, GDR)\footnote{1point5
refers to the warming limit of the Paris accord} 
gathering an interdisciplinary team of engineers and researchers from various research fields in France. 
One of the main outputs of this collaboration is the \emph{GES 1point5},\footnote{GES
is the French acronym for greenhouse gases.} a
standardized online tool for the accounting of the carbon 
footprint of French research labs. What follows is a brief
summary of the latter. We refer to \cite{labos1p5}
for a publication describing it. Further information 
can be found on the website of the labos1point5 collaboration 
\cite{labos1p5web}. The tool itself is available at \cite{ges1p5}, 
while the open source code is hosted at \cite{ges1p5git}.
The GDR labos1point5 is also active in helping labs in their
transition to a lower footprint, in developing new ways of 
teaching climate and ecological aspects to students, as well
as of communicating to the general public. Finally,
there are  teams dedicated to the reflection on the role
of science in the climate crisis and on fostering collaboration 
between arts and science in this context.

As explained in \cite{labos1p5}, one of the main motivations for the 
creation of the GES 1point5 tool was the difficulty in aggregating or comparing the results of 
the many existing studies in the literature on the carbon footprint of academic research.
This difficulty was caused by the sensitivity of the footprint to the applied 
methodology, which made comparisons extremely challenging, since discrepancies in results
could not be disentangled from methodological differences. 
The creation of the GES 1point5 was then intended to provide 
a tool specifically designed to estimate the carbon footprint of research with a transparent
and accessible methodology and a database of carbon footprints assessed with the same
methodology to enable a robust comparison of research carbon footprints across institutions,
contexts or disciplines.

More specifically, GES 1point5 allows research labs to
estimate their yearly emissions --- as of February 15, 2023,
628 laboratories have compiled 1140 yearly carbon footprint 
determinations \cite{labos1p5web}.
Currently, GES 1point5 can estimate GHG emissions 
due to the energy consumption and refrigerant gases of the labs' buildings, 
those attributed to the purchase of their digital devices and to computing, 
commuting, professional travel as well as the associated uncertainties.
A module estimating emissions from consumables' purchases has recently been added.
In all cases, the estimation is based upon its established
standardised methodology and the database of emission factors and turns out 
to be relatively straightforward for the end user.\footnote{In the case
of professional travel, a feature has been added to the internal management software
of CNRS that can output the travel data --- origin, destination, means of transportation, etc... ---
for a given year in a ready-for-GES 1point5 format,
sparing the end user tedious data entry/conversion.} To give an example, 
commuting emissions are estimated by gathering data through an anonymised survey
sent to staff members, which can be answered in less than five minutes.
For each specified commute (up to two different ones per week can be entered)
GES 1point5 multiplies the distance traveled via each means of transportation times
the specific emission factor from its database. These are collected in \fref{fig:emiMobility} in the section on Mobility.
The underlying routines are available at \cite{1p5commute} for everyone to test their 
commutes; similarly, those used in the determination of professional travel emissions are 
available at \cite{1p5travel} and those for purchases at 
\cite{1p5purchases}.

Once the emissions have been estimated, GES 1point5 presents the results 
through its graphic interface, highlighting the main drivers of the carbon footprint. 
Finally, emission reduction actions that may be undertaken after the evaluation of the 
footprint can be evaluated in the subsequent years, thanks to the reliance
on a standardized protocol. The latter also allows the aforementioned 
aggregation and comparison of GES 1point5-based carbon footprints.

As stated in \cite{labos1p5}, while some aspects of GES 1point5 are specific 
to the context of French research, the tool  may be reused in research centers elsewhere, provided
the necessary adjustments --- e.g. carbon intensity of the grid --- are taken. 
A French and an English language version of GES 1point5 are built-in in the current version to ease deployment in any country; contacts with several institutions outside France 
have been established.
\end{bestpractice}

\end{document}




