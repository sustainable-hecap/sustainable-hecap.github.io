\documentclass[../SustainableHEP.tex]{subfiles}
\begin{document}
\RaggedRight
\sloppy
\newpage

\section*{Forewords}
\label{sec:Forewords}
\addcontentsline{toc}{section}{Forewords}

\textcolor{Pythongreen}{\rule{2cm}{3pt}}

\begin{quotation}
In the past century the ever increasing resource demands of humans have a devastating impact on the climate of our planet. The resulting heat waves, droughts, strong rain falls, violent storms, melting ice and rising sea levels are posing an existential threat to many people world-wide already today, and even more in the future. This situation demands action from all of us individually, and as groups and institutions, to do whatever we can to reduce (or ideally eliminate) the emission of \CdO  (equivalent) gases, and more generally to preserve the resources of the planet. This report illustrates well that scientists working in the fields of high energy physics, cosmology and astroparticle physics are using on average more resources than what will be acceptable, and action is needed immediately. Many recommendations are proposed for individuals, groups and/or institutions. Some are rather easy while others are challenging and require core habits to change. Given the very international nature of the research performed in these fields, there is a large potential to propagate the actions to 100s of institutions in countries of all continents and thereby increasing the impact further. I very much hope this document will help us embark on the right path.
\end{quotation}
\begin{flushleft}
Prof.~Dr.~Beate Heinemann\\
Director for Particle Physics at DESY, Professor of Physics at Albert-Ludwigs-Universitaet Freiburg, and former Deputy Spokesperson for the ATLAS Collaboration at CERN
\end{flushleft}

\textcolor{Pythongreen}{\rule{2cm}{3pt}}

\begin{quotation}
Climate change does not necessarily threaten the survival of our planet in the bigger picture of the solar system and the universe; rather, it threatens our own survival and the survival of the bio- and eco-spheres that we rely on for our subsistence in this cosmos. We have no alternative planet which will provide for us, so we better not make our current one uninhabitable.
Climate research has been clear on the effects of climate change, and slowly the rising sea levels, the disappearance of glaciers, the frequency of hundred year floods, the storms, droughts, and heat waves beat the message home: climate change is happening, human action is the cause, and we better counter-act it immediately.
As fellow scientists from the areas of high energy physics, cosmology, and astroparticle physics, we are trained to understand and consider scientific results in our daily work. We know how to interpret statistics and draw conclusions from data. 
This document is a start to take the conclusions from climate research seriously and put them into action in our own fields of research. Collecting and reflecting on available results, together with recommendations for the implementation --- from easy to hard --- is an important first step, but only a first step. Let's use this document and get started in transforming our field of research into a sustainable field of research - for the benefit of our planet and our own futures. The data has been clear for a while, now is the time to act on it.
\end{quotation}
\begin{flushleft}
Dr.~Valerie Lang \\
Chair of the management board of the young High Energy Physicists (yHEP) association, Germany,
Researcher at the Albert-Ludwigs-Universitaet Freiburg, Germany, and
Member of the ATLAS Collaboration at CERN
\end{flushleft}

\textcolor{Pythongreen}{\rule{2cm}{3pt}}

\end{document}