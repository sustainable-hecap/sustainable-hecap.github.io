\documentclass[../SustainableHEP.tex]{subfiles}
\graphicspath{{\subfix{Sections/Figs/}}}
\begin{document}
\RaggedRight
\sloppy
\begin{titlepage}

    \doublespacing    
    \begin{flushleft}
    % Placeholder title to be discussed
    \textbf{\huge Environmental sustainability in basic research}\\
    \textbf{\Large A perspective from HECAP+}
    %\textbf{\Large Striving towards Environmental Sustainability in High Energy Physics, Cosmology and Astroparticle Physics (HECAP), and Hadron and Nuclear Physics}
    \end{flushleft}
    \singlespacing

    \vspace{27em}

    \noindent {\bf Abstract}\\

    \noindent The climate crisis and the degradation of the world's ecosystems require humanity to take immediate action. The international scientific community has a responsibility to limit the negative environmental impacts of its research. The HECAP+ communities (High Energy Physics, Cosmology, Astroparticle Physics, plus Hadron and Nuclear Physics) make use of common and similar experimental infrastructure, such as accelerators, and rely similarly on the processing of big data. Our communities therefore face similar challenges to improving the sustainability of our research. This document aims to reflect on the environmental impacts of our work practices and research infrastructure, to highlight best practice, to make recommendations for positive changes, and to identify the opportunities and challenges that such changes present for wider aspects of social responsibility.\\
           
    \begin{flushright}
        \textbf{Version:\ Draft, May 2023}\\
        \textbf{\textcolor{Pythongreen}{Please read this document in electronic format where possible and refrain from printing it unless absolutely necessary. Thank you.}}
    \end{flushright}

\end{titlepage}

\newpage

\thispagestyle{empty}

~

\vspace{23em}
\RaggedRight

\noindent This document has been typeset in LaTeX using Atkinson Hyperlegible to maximise readability (see \url{https://tug.org/FontCatalogue/atkinsonhyperlegible/}).

\noindent An HTML version of this document is available at:~\url{https://sustainable-hecap.github.io/}.

\newpage

\thispagestyle{empty}

\section*{Statement of intent}
    
        This reflective document was developed as part of a grassroots initiative {\it Striving towards Environmental Sustainability in High Energy Physics, Cosmology and Astroparticle Physics}.
        It is intended to be a synthesis of current data and best practices from research in climate science and sustainability, as applied to our fields to the best of our ability as physicists, and a reflection on the roles that our communities can play in limiting negative environmental impacts due to our research work and scientific culture. The focus is not to stipulate the research that our communities should undertake.
        
        The scope is inspired by the holistic approach of annual environmental reports of major institutes, which include emissions directly related to research and collateral emissions, such as from personal commutes and institutional catering.
        Continuing to address this broad scope will require input from across the community, in particular to identify the technical challenges of limiting the environmental impacts of our current and future research infrastructure.

        While this document is primarily framed from the perspective of HECAP+, much of its discussion applies to basic research more generally. It is intended to be a first step, and it is hoped that it may serve as an example to ours and other fields.
        
        Comments on this document are welcome. Please get in touch with us via the online platform at:~\url{https://sustainable-hecap.github.io/}, where individual endorsement of this document can also be made.
        
        \noindent \textbf{Thank you for taking the time to read this document.}
        
\end{document}